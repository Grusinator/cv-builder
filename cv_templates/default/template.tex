%%%%%%%%%%%%%%%%%%%%%%%%%%%%%%%%%%%%%%%%%
% Twenty Seconds Resume/CV
% LaTeX Template
% Version 1.0 (14/7/16)
%
% Original author:
% Carmine Spagnuolo (cspagnuolo@unisa.it) with major modifications by 
% Vel (vel@LaTeXTemplates.com) and Harsh (harsh.gadgil@gmail.com)
%
% License:
% The MIT License (see included LICENSE file)
%
%%%%%%%%%%%%%%%%%%%%%%%%%%%%%%%%%%%%%%%%%

%----------------------------------------------------------------------------------------
%	PACKAGES AND OTHER DOCUMENT CONFIGURATIONS
%----------------------------------------------------------------------------------------

\documentclass[letterpaper]{twentysecondcv} % a4paper for A4

% Command for printing skill overview bubbles
\newcommand\skills{ 
~
	\smartdiagram[bubble diagram]{
        \textbf{Data}\\\textbf{Engineering},
        \textbf{Databricks},
        \textbf{Software}\\\textbf{Dev.},
        \textbf{Machine}\\\textbf{Learning},
        \textbf{DevOps},
        \textbf{~~Azure~~}
    }
}

% Programming skill bars
\programming{{ADF / 2.5}, {SQL / 3}, {pyspark / 3.5}, {Python / 5}}

%}


%----------------------------------------------------------------------------------------
%	 PERSONAL INFORMATION
%----------------------------------------------------------------------------------------

\cvname{William Sandvej Hansen} % Your name
\cvjobtitle{ Data Engineer } % Job
% title/career

\cvlinkedin{/in/william-sandvej-hansen}
\cvgithub{grusinator}
\cvnumberphone{+45 40371757} % Phone number
\cvsite{pro-solution.dk} % Personal website
\cvmail{grusinator@gmail.com} % Email address

%----------------------------------------------------------------------------------------

\begin{document}

\makeprofile % Print the sidebar


%----------------------------------------------------------------------------------------
%	 EXPERIENCE
%----------------------------------------------------------------------------------------

\section{Experience}

\begin{twenty} % Environment for a list with descriptions

\twentyitem
    	{dec22 - okt23}
		{}
        {Data Engineer (Freelance)}
        {\href{https://www.energinet.dk/}{Energinet}}
        {}
        {
        \begin{itemize}
            \item Orchestrated a data project at Energinet to manage extensive datasets from Denmark's energy island, ensuring data reception, quality control, and distribution to stakeholders, using Azure Data factory and Blob Storage.
            \item Utilized Spark and Databricks for scalable processing of diverse binary offshore sensor data, accommodating various formats like GDB, DFSU, segy, and xtf.
            \item Implemented Sedona (GeoSpark) within Spark to support geospatial data, enabling efficient storage in Delta Parquet format and optimizing read performance through spatial partitioning with geohashing.
\end{itemize}
\bigskip % Add a big skip for more space
}
 
\twentyitem
    	{mar20 - nov22}
		{}
        { Data Engineer}
        {\href{https://orsted.com/}{Ørsted}}
        {}
        {
          \begin{itemize}
    \item Developed a Python-based data validation component within data pipelines, utilizing Pandas and integrated with Azure Service Bus.
    \item Engaged in data modeling with MS SQL Server and REST APIs to enhance data accessibility for analytics tools.
    \item Streamlined database schema iteration using SQLAlchemy and SQLite, simulating the production environment for reliable pre-deployment testing.
    \item Crafted analytics and visualization tools with Streamlit, Panel, Dash, and Bokeh, enabling engineers to derive insights from data.
\end{itemize}
\bigskip % Add a big skip for more space
}

\twentyitem
    {nov18 - nov19}
    {}
    {IT Consultant}
    {\href{https://www.netcompany.com/}{Netcompany}}
    {}
    {
        \begin{itemize}
            \item Collaborated on key projects, contributing to software development and systems integration.
        \end{itemize}
      \bigskip % Add a big skip for more space
    }

        

\twentyitem
    	{sep16 - mar18}
		{}
        { Software Developer}
        {\href{https://www.niras.dk/ydelser/gis-geodata-og-automation/}{Niras}}
        {}
        {
            \begin{itemize}
                \item Processed geospatial data using PostgreSQL and Pandas, and applied machine learning for image classification tasks.
                \item Employed CNNs to classify buildings in orthophotos, aiding a national geocoding project's quality assurance.
                \item Designed and developed a pipeline to convert point cloud data into actionable insights for infrastructure maintenance.
                \item Created algorithms for detecting and classifying bird species from aerial imagery to analyze behavioral patterns.
                \break
            \end{itemize}
        }
   

\end{twenty}


%----------------------------------------------------------------------------------------
%	 EDUCATION
%----------------------------------------------------------------------------------------
\section{Education}

\begin{twenty} % Environment for a list with descriptions
	\twentyitem
    	{2014 - 2016}
        {}
        {MSc., Earth and Space Physics and Engineering}
        {\href{http://www.dtu.dk/english/education/msc/programmes/earth_and_space_physics_and_engineering}{DTU}}
        {}
        {
        remote sensing, photogrammetry, image processing and GPS.
        Spatial and temporal storage and analysis of data using databases and statistical modelling and machine learning.
        \bigskip % Add a big skip for more space
        }
        
        
	\twentyitem
    	{2010 - 2014}
		{}
        {BEng., Electrical Engineering}
        {\href{http://sdu.dk}{SDU}}
        {}
        {
            Electrical signal processing and HF circuit design 
        }
	
\end{twenty}

\section{Other Projects}
{
    \begin{itemize}
        \item \textbf{Dansk Miljø Analyse:} Deep learning based waveform classification on chemical spectrograms in order to determine concentrations of specific compounds in samples.
    \end{itemize}}

\section{Courses}
{
    \begin{itemize}
        \item IDA Big Data Course: Hands on with Hadoop and Spark.
    \end{itemize}}

    \section{Competencies}

    \begin{table}[h]
        \centering
        \begin{tabular}{|l|l|l|l|}
        \hline
        \textbf{Competency} & \textbf{Last Used} & \textbf{Years of Experience} & \textbf{Rating (1-5)} \\
        \hline
        Data Engineering & 2022 & 3 & 4 \\
        \hline
        Databricks & 2021 & 2 & 3 \\
        \hline
        Software Development & 2022 & 5 & 5 \\
        \hline
        Machine Learning & 2020 & 4 & 4 \\
        \hline
        DevOps & 2021 & 3 & 3 \\
        \hline
        Azure & 2022 & 4 & 4 \\
        \hline
        \end{tabular}
    \end{table}


\end{document} 