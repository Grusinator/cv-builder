%%%%%%%%%%%%%%%%%
% This is an sample CV template created using altacv.cls
% (v1.7, 9 August 2023) written by LianTze Lim (liantze@gmail.com). Compiles with pdfLaTeX, XeLaTeX and LuaLaTeX.
%
%% It may be distributed and/or modified under the
%% conditions of the LaTeX Project Public License, either version 1.3
%% of this license or (at your option) any later version.
%% The latest version of this license is in
%%    http://www.latex-project.org/lppl.txt
%% and version 1.3 or later is part of all distributions of LaTeX
%% version 2003/12/01 or later.
%%%%%%%%%%%%%%%%

%% Use the "normalphoto" option if you want a normal photo instead of cropped to a circle
% \documentclass[10pt,a4paper,normalphoto]{altacv}

\documentclass[10pt,a4paper,ragged2e,withhyper]{altacv}
%% AltaCV uses the fontawesome5 and packages.
%% See http://texdoc.net/pkg/fontawesome5 for full list of symbols.

% Change the page layout if you need to
\geometry{left=1.25cm,right=1.25cm,top=1.5cm,bottom=1.5cm,columnsep=1.2cm}

% The paracol package lets you typeset columns of text in parallel
\usepackage{paracol}

% Change the font if you want to, depending on whether
% you're using pdflatex or xelatex/lualatex
% WHEN COMPILING WITH XELATEX PLEASE USE
% xelatex -shell-escape -output-driver="xdvipdfmx -z 0" sample.tex
\ifxetexorluatex
  % If using xelatex or lualatex:
  \setmainfont{Roboto Slab}
  \setsansfont{Lato}
  \renewcommand{\familydefault}{\sfdefault}
\else
  % If using pdflatex:
  \usepackage[rm]{roboto}
  \usepackage[defaultsans]{lato}
  % \usepackage{sourcesanspro}
  \renewcommand{\familydefault}{\sfdefault}
\fi

% Change the colours if you want to
\definecolor{SlateGrey}{HTML}{2E2E2E}
\definecolor{LightGrey}{HTML}{666666}
\definecolor{DarkPastelRed}{HTML}{450808}
\definecolor{PastelRed}{HTML}{8F0D0D}
\definecolor{GoldenEarth}{HTML}{E7D192}
\colorlet{name}{black}
\colorlet{tagline}{PastelRed}
\colorlet{heading}{DarkPastelRed}
\colorlet{headingrule}{GoldenEarth}
\colorlet{subheading}{PastelRed}
\colorlet{accent}{PastelRed}
\colorlet{emphasis}{SlateGrey}
\colorlet{body}{LightGrey}

% Change some fonts, if necessary
\renewcommand{\namefont}{\Huge\rmfamily\bfseries}
\renewcommand{\personalinfofont}{\footnotesize}
\renewcommand{\cvsectionfont}{\LARGE\rmfamily\bfseries}
\renewcommand{\cvsubsectionfont}{\large\bfseries}


% Change the bullets for itemize and rating marker
% for \cvskill if you want to
\renewcommand{\cvItemMarker}{{\small\textbullet}}
\renewcommand{\cvRatingMarker}{\faCircle}
% ...and the markers for the date/location for \cvevent
% \renewcommand{\cvDateMarker}{\faCalendar*[regular]}
% \renewcommand{\cvLocationMarker}{\faMapMarker*}


% If your CV/résumé is in a language other than English,
% then you probably want to change these so that when you
% copy-paste from the PDF or run pdftotext, the location
% and date marker icons for \cvevent will paste as correct
% translations. For example Spanish:
% \renewcommand{\locationname}{Ubicación}
% \renewcommand{\datename}{Fecha}


%% Use (and optionally edit if necessary) this .tex if you
%% want to use an author-year reference style like APA(6)
%% for your publication list
% % When using APA6 if you need more author names to be listed
% because you're e.g. the 12th author, add apamaxprtauth=12
\usepackage[backend=biber,style=apa6,sorting=ydnt]{biblatex}
\defbibheading{pubtype}{\cvsubsection{#1}}
\renewcommand{\bibsetup}{\vspace*{-\baselineskip}}
\AtEveryBibitem{%
  \makebox[\bibhang][l]{\itemmarker}%
  \iffieldundef{doi}{}{\clearfield{url}}%
}
\setlength{\bibitemsep}{0.25\baselineskip}
\setlength{\bibhang}{1.25em}


%% Use (and optionally edit if necessary) this .tex if you
%% want an originally numerical reference style like IEEE
%% for your publication list
\usepackage[backend=biber,style=ieee,sorting=ydnt,defernumbers=true]{biblatex}
%% For removing numbering entirely when using a numeric style
\setlength{\bibhang}{1.25em}
\DeclareFieldFormat{labelnumberwidth}{\makebox[\bibhang][l]{\itemmarker}}
\setlength{\biblabelsep}{0pt}
\defbibheading{pubtype}{\cvsubsection{#1}}
\renewcommand{\bibsetup}{\vspace*{-\baselineskip}}
\AtEveryBibitem{%
  \iffieldundef{doi}{}{\clearfield{url}}%
}


%% sample.bib contains your publications
\addbibresource{sample.bib}

\begin{document}
\name{William Sandvej Hansen}
\tagline{Data Engineer}
%% You can add multiple photos on the left or right
\photoR{2.8cm}{images/william1}
% \photoL{2.5cm}{Yacht_High,Suitcase_High}

\personalinfo{%
  % Not all of these are required!
  \email{grusinator@gmail.com}
  \phone{40371757}
  \mailaddress{Sømose Hegn 78, st. 2, 2750 Ballerup}
  \location{Capital Region of Denmark, Denmark}
  % \homepage{www.linkedin.com/in/william-sandvej-hansen}
  \linkedin{william-sandvej-hansen}
  \github{grusinator}
  %% You can add your own arbitrary detail with
  %% \printinfo{symbol}{detail}[optional hyperlink prefix]
  % \printinfo{\faPaw}{Hey ho!}[https://example.com/]

  %% Or you can declare your own field with
  %% \NewInfoFiled{fieldname}{symbol}[optional hyperlink prefix] and use it:
  % \NewInfoField{gitlab}{\faGitlab}[https://gitlab.com/]
  % \gitlab{your_id}
  %%
  %% For services and platforms like Mastodon where there isn't a
  %% straightforward relation between the user ID/nickname and the hyperlink,
  %% you can use \printinfo directly e.g.
  % \printinfo{\faMastodon}{@username@instace}[https://instance.url/@username]
  %% But if you absolutely want to create new dedicated info fields for
  %% such platforms, then use \NewInfoField* with a star:
  % \NewInfoField*{mastodon}{\faMastodon}
  %% then you can use \mastodon, with TWO arguments where the 2nd argument is
  %% the full hyperlink.
  % \mastodon{@username@instance}{https://instance.url/@username}
}

\makecvheader
%% Depending on your tastes, you may want to make fonts of itemize environments slightly smaller
% \AtBeginEnvironment{itemize}{\small}

%% Set the left/right column width ratio to 6:4.
\columnratio{0.6}

% Start a 2-column paracol. Both the left and right columns will automatically
% break across pages if things get too long.
\begin{paracol}{2}


\cvsubsection{Summary}

% \cvevent{}{}{}{}
I have been working on various projects ranging from backenddevelopment to data engineering to data analysis. 
Mostly withinengineering and energy sectors. 
I have a bit of GIS experience fromvarious projects, well supported by my education in data processingwithin geospatial and remote sensing diciplines. 
Now adays its hardto not get a bit of cloud and devops under the skin as well. 
I am quite adaptive in terms of skills and competences, and not afraid to start out something that i might not know much about. 
In terms of personality, according to the "Insights" profile test, i am the "Helping Inspirer". 
creative, energetic and currious type. Sometimes a bit too quick and intuitive. But it has its qualities too.


\cvsection{Experience}

\cvevent{Data Engineer}{Energinet}{December 2022 -- October 2023}{Copenhagen, Denmark}
\begin{itemize}
\item Developed a data project collecting massive amounts of data from the energy island in Denmark, from various providers.
\item Performed quality control and delivered data to developers.
\item Utilized Spark and Databricks for scalability and flexibility in handling various data formats.
\item Used Sedona (GeoSpark) for geospatial data processing and storage in Delta Parquet format.
\item Implemented spatial partitioning using geohashing for improved read performance.
\end{itemize}

\divider

\cvevent{Data Engineer}{Ørsted}{March 2020 -- November 2022}{Copenhagen, Denmark}
\begin{itemize}
\item Developed a data validation component for data pipelines using Python and Pandas.
\item Worked on data modeling to provide easy-to-understand and consume data for analytics tools.
\item Developed data analytics and visualization tools using Python libraries such as Streamlit, Panel, Dash, and Bokeh.
\item Used MS SQL Server, REST API, and Azure Service Bus for integration and data processing.
\item Implemented a mock database using Python, SQLAlchemy, and SQLite for testing changes before production implementation.
\item Used SAFE framework for project management and Azure DevOps, Docker, and Kubernetes for DevOps.
\end{itemize}

\divider

\cvevent{IT Consultant}{Netcompany}{November 2018 -- November 2019}{Copenhagen, Denmark}
\begin{itemize}
\item Worked on building custom IT solutions with multiple integrations using Oracle, Groovy, REST, and JavaScript.
\item Used Jira and Git with SCRUM for project management and version control.
\end{itemize}

\divider

\cvevent{Software Developer}{NIRAS}{September 2016 -- March 2018}{Allerød, Denmark}
\begin{itemize}
\item Developed geodata algorithms for data transformation, including lidar data processing and image classification.
\item Developed plugins for QGIS using Qt and Python.
\item Used C\# and Postgres/PostGIS for developing models and processing geospatial data.
\end{itemize}

% \divider

% \cvevent{Engineer}{KK Wind Solutions}{March 2014 -- August 2014}{Ikast, Denmark}
% \begin{itemize}
% \item Investigated the effort of switching out a communication chip for the IO boards in the wind turbine control system.
% \item Programmed the chip in C and investigated communication protocols such as Ethercat and Profinet.
% \end{itemize}

% \divider

% \cvevent{Instructor}{Det Tekniske Fakultet, Syddansk Universitet}{September 2013 -- January 2014}{Odense, Denmark}
% \begin{itemize}
% \item Taught basic analog circuit design in electronics courses.
% \end{itemize}

% \divider

% \cvevent{Internship Electronics}{VELUX}{February 2013 -- June 2013}{Skjern, Denmark}
% \begin{itemize}
% \item Developed a PCB for testing the durability of solar panels for automated Velux windows.
% \end{itemize}

% \cvsection{Projects}

% \cvevent{Project 1}{Funding agency/institution}{}{}
% \begin{itemize}
% \item Details
% \end{itemize}

% \divider

% \cvevent{Project 2}{Funding agency/institution}{Project duration}{}
% A short abstract would also work.

% \medskip

% \cvsection{A Day of My Life}

% % Adapted from @Jake's answer from http://tex.stackexchange.com/a/82729/226
% % \wheelchart{outer radius}{inner radius}{
% % comma-separated list of value/text width/color/detail}
% \wheelchart{1.5cm}{0.5cm}{%
%   6/8em/accent!30/{Sleep,\\beautiful sleep},
%   3/8em/accent!40/Hopeful novelist by night,
%   8/8em/accent!60/Daytime job,
%   2/10em/accent/Sports and relaxation,
%   5/6em/accent!20/Spending time with family
% }

% use ONLY \newpage if you want to force a page break for
% ONLY the current column
\newpage

% \cvsection{Publications}

% %% Specify your last name(s) and first name(s) as given in the .bib to automatically bold your own name in the publications list.
% %% One caveat: You need to write \bibnamedelima where there's a space in your name for this to work properly; or write \bibnamedelimi if you use initials in the .bib
% %% You can specify multiple names, especially if you have changed your name or if you need to highlight multiple authors.
% \mynames{Lim/Lian\bibnamedelima Tze,
%   Wong/Lian\bibnamedelima Tze,
%   Lim/Tracy,
%   Lim/L.\bibnamedelimi T.}
% %% MAKE SURE THERE IS NO SPACE AFTER THE FINAL NAME IN YOUR \mynames LIST

% \nocite{*}

% \printbibliography[heading=pubtype,title={\printinfo{\faBook}{Books}},type=book]

% \divider

% \printbibliography[heading=pubtype,title={\printinfo{\faFile*[regular]}{Journal Articles}},type=article]

% \divider

% \printbibliography[heading=pubtype,title={\printinfo{\faUsers}{Conference Proceedings}},type=inproceedings]

%% Switch to the right column. This will now automatically move to the second
%% page if the content is too long.


\cvsection{Skill Matrix}
\begin{tabular}{|l|l|l|l|}
\hline
Working Area & Level & Last Used & Years of exp \\
\hline
Full Stack & Experienced & 2022 & 3 \\
Frontend & Experienced & 2022 & 2 \\
Data Modeling & Highly experienced & 2022 & 4 \\
Data Transformation & Highly experienced & 2022 & 4 \\
Data Science & Experienced & 2022 & 2 \\
Big Data & & 2023 & 1 \\
Unit Testing & Highly experienced & 2023 & 4 \\
PostgreSQL & Experienced & 2020 & 2 \\
Oracle & Knowledgeable & 2019 & 1 \\
Microsoft SQL Server & Experienced & 2023 & 2 \\
SQLite & Highly experienced & 2022 & 3 \\
% PostGIS & Experienced & 2018 & 2 \\
% Kubernetes & Experienced & 2022 & 2 \\
% Atlassian Jira & Knowledgeable & 2019 & 2 \\
% Git & Highly experienced & 2023 & 5 \\
% Renewable Energy & Experienced & 2022 & 4 \\
% Counselling & Experienced & 2023 & 3 \\
% Electronics & Knowledgeable & 2014 & 1 \\
% Offshore Wind Energy & Highly experienced & 2023 & 4 \\
% Data Engineering & Highly experienced & 2023 & 5 \\
% DevOps & Highly experienced & 2023 & 3 \\
% Data Visualization & Experienced & 2023 & 4 \\
% SAFe & Highly experienced & 2019 & 3 \\
% Scrum & Highly experienced & 2023 & 4 \\
% Agile Software Development & Highly experienced & 2023 & 4 \\
% Web Scraping & Knowledgeable & 2019 & 1 \\
% Containerization & Knowledgeable & 2022 & 3 \\
% CI/CD & Highly experienced & 2023 & 4 \\
% .NET & Experienced & 2020 & 2 \\
% Microsoft Azure & Highly experienced & 2023 & 3 \\
% Linux & Experienced & 2022 & 4 \\
% Xamarin & Knowledgeable & 2018 & 1 \\
% AWS & Some knowledge & 2018 & 1 \\
% Python & Expert & 2023 & 6 \\
% C\# & Highly experienced & 2020 & 2 \\
% SQL & Highly experienced & 2023 & 4 \\
% Groovy & Experienced & 2019 & 1 \\
% JavaScript & Experienced & 2019 & 2 \\
% Geographical Information Systems (GIS) & Highly experienced & 2023 & 2 \\
% Machine Learning (ML) & Experienced & 2018 & 2 \\
% Docker & Experienced & 2022 & 2 \\
% Microsoft Azure Data Factory & Highly experienced & 2023 & 1 \\
% Flask & Experienced & 2019 & 1 \\
% REST & Highly experienced & 2022 & 4 \\
% Django & Experienced & 2020 & 2 \\
% Amazon S3 & Experienced & 2020 & 1 \\
% Spark & Highly experienced & 2023 & 2 \\
% Microsoft Azure Service Bus & Experienced & 2022 & 2 \\
% Pandas (Python) & Highly experienced & 2022 & 5 \\
% React & Some knowledge & 2019 & 1 \\
% Svelte & Some knowledge & 2023 & 1 \\
\hline
\end{tabular}



\switchcolumn

% \cvsection{My Life Philosophy}

% \begin{quote}
% ``Something smart or heartfelt, preferably in one sentence.''
% \end{quote}

% \cvsection{Most Proud of}

% \cvachievement{\faTrophy}{Fantastic Achievement}{and some details about it}

% \divider

% \cvachievement{\faHeartbeat}{Another achievement}{more details about it of course}

% \divider

% \cvachievement{\faHeartbeat}{Another achievement}{more details about it of course}


% Languages

\cvsection{Languages}

\cvskill{Python}{6}
\cvskill{SQL}{4}
\cvskill{JavaScript}{3}
\cvskill{C}{3} %% Supports X.5 values.
% \cvskill{Groovy}{2}

% Skills

\cvsection{Skills}

\cvsubsection{Technologies}

\cvtag{PostgreSQL}
\cvtag{Oracle}
\cvtag{Microsoft SQL Server}
\cvtag{Kubernetes}
\cvtag{Git}
% \cvtag{DevOps}
% \cvtag{Data Visualization}
% \cvtag{Scrum}
% \cvtag{Agile Software Development}
% \cvtag{Containerization}
% \cvtag{CI/CD}
% \cvtag{.NET}
% \cvtag{Microsoft Azure}
% \cvtag{AWS}
% \cvtag{Machine Learning (ML)}
% \cvtag{Docker}
% \cvtag{Microsoft Azure Data Factory}
% \cvtag{REST}
% \cvtag{Django}
% \cvtag{Spark}
% \cvtag{Pandas (Python)}

\cvsubsection{Software Development}

\cvtag{Full Stack}
\cvtag{Frontend}
\cvtag{Data Modeling}
\cvtag{Data Transformation}
\cvtag{Data Science}
\cvtag{Big Data}
\cvtag{Unit Testing}


\cvsubsection{Databases}

\cvtag{PostgreSQL}
\cvtag{Oracle}
\cvtag{Microsoft SQL Server}
\cvtag{SQLite}
\cvtag{PostGIS}


\cvsubsection{Development Tools}

\cvtag{Kubernetes}
\cvtag{Atlassian Jira}
\cvtag{Git}
\cvtag{Containerization}
\cvtag{CI/CD}


\cvsubsection{Industries}

\cvtag{Renewable Energy}
\cvtag{Counselling}
\cvtag{Electronics}
\cvtag{Offshore Wind Energy}


\cvsubsection{Methodologies}

\cvtag{SAFe}
\cvtag{Scrum}
\cvtag{Agile Software Development}


\cvsubsection{Platforms}

\cvtag{.NET}
\cvtag{Linux}
\cvtag{Xamarin}
\cvtag{Microsoft Azure}
\cvtag{AWS}


\cvsubsection{Technologies}

\cvtag{Machine Learning (ML)}
\cvtag{Docker}
\cvtag{Azure Data Factory}
\cvtag{Flask}
\cvtag{REST}
\cvtag{Django}
\cvtag{Amazon S3}
\cvtag{Spark}
\cvtag{Azure Service Bus}
\cvtag{Pandas (Python)}
\cvtag{React}
\cvtag{Svelte}


\cvsection{Traits}

\cvtag{Motivator \& Leader}




%% Yeah I didn't spend too much time making all the
%% spacing consistent... sorry. Use \smallskip, \medskip,
%% \bigskip, \vspace etc to make adjustments.
\medskip

\cvsection{Education}

\cvevent{Master of Engineering - MEng, Earth and Space Physics and Engineering}{Danmarks Tekniske Universitet}{2014 -- 2016}{}
\divider

\cvevent{Bachelor of Engineering (BEng), Electrical and Electronics Engineering}{Syddansk Universitet}{2010 -- 2014}{}
\divider

%  \cvevent{HTX, Science - Mathematics and Physics}{Vejle Tekniske Gymnasium}{2007 -- 2010}{}

% \divider

% \cvsection{Referees}

% % \cvref{name}{email}{mailing address}
% \cvref{Prof.\ Alpha Beta}{Institute}{a.beta@university.edu}
% {Address Line 1\\Address line 2}

% \divider

% \cvref{Prof.\ Gamma Delta}{Institute}{g.delta@university.edu}
% {Address Line 1\\Address line 2}


\end{paracol}
\end{document}
