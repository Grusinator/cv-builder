%%%%%%%%%%%%%%%%%
% This is an sample CV template created using altacv.cls
% (v1.7, 9 August 2023) written by LianTze Lim (liantze@gmail.com). Compiles with pdfLaTeX, XeLaTeX and LuaLaTeX.
%
%% It may be distributed and/or modified under the
%% conditions of the LaTeX Project Public License, either version 1.3
%% of this license or (at your option) any later version.
%% The latest version of this license is in
%%    http://www.latex-project.org/lppl.txt
%% and version 1.3 or later is part of all distributions of LaTeX
%% version 2003/12/01 or later.
%%%%%%%%%%%%%%%%

%% Use the "normalphoto" option if you want a normal photo instead of cropped to a circle
% \documentclass[10pt,a4paper,normalphoto]{altacv}

\documentclass[10pt,a4paper,ragged2e,withhyper]{altacv}
%% AltaCV uses the fontawesome5 and packages.
%% See http://texdoc.net/pkg/fontawesome5 for full list of symbols.

% Change the page layout if you need to
\geometry{left=1.25cm,right=1.25cm,top=1.5cm,bottom=1.5cm,columnsep=1.2cm}

% The paracol package lets you typeset columns of text in parallel
\usepackage{paracol}

% Change the font if you want to, depending on whether
% you're using pdflatex or xelatex/lualatex
% WHEN COMPILING WITH XELATEX PLEASE USE
% xelatex -shell-escape -output-driver="xdvipdfmx -z 0" sample.tex
\ifxetexorluatex
  % If using xelatex or lualatex:
  \setmainfont{Roboto Slab}
  \setsansfont{Lato}
  \renewcommand{\familydefault}{\sfdefault}
\else
  % If using pdflatex:
  \usepackage[rm]{roboto}
  \usepackage[defaultsans]{lato}
  % \usepackage{sourcesanspro}
  \renewcommand{\familydefault}{\sfdefault}
\fi

% Change the colours if you want to
\definecolor{SlateGrey}{HTML}{2E2E2E}
\definecolor{LightGrey}{HTML}{666666}
\definecolor{DarkPastelRed}{HTML}{450808}
\definecolor{PastelRed}{HTML}{8F0D0D}
\definecolor{GoldenEarth}{HTML}{E7D192}
\colorlet{name}{black}
\colorlet{tagline}{PastelRed}
\colorlet{heading}{DarkPastelRed}
\colorlet{headingrule}{GoldenEarth}
\colorlet{subheading}{PastelRed}
\colorlet{accent}{PastelRed}
\colorlet{emphasis}{SlateGrey}
\colorlet{body}{LightGrey}

% Change some fonts, if necessary
\renewcommand{\namefont}{\Huge\rmfamily\bfseries}
\renewcommand{\personalinfofont}{\footnotesize}
\renewcommand{\cvsectionfont}{\LARGE\rmfamily\bfseries}
\renewcommand{\cvsubsectionfont}{\large\bfseries}


% Change the bullets for itemize and rating marker
% for \cvskill if you want to
\renewcommand{\cvItemMarker}{{\small\textbullet}}
\renewcommand{\cvRatingMarker}{\faCircle}
% ...and the markers for the date/location for \cvevent
% \renewcommand{\cvDateMarker}{\faCalendar*[regular]}
% \renewcommand{\cvLocationMarker}{\faMapMarker*}


% If your CV/résumé is in a language other than English,
% then you probably want to change these so that when you
% copy-paste from the PDF or run pdftotext, the location
% and date marker icons for \cvevent will paste as correct
% translations. For example Spanish:
% \renewcommand{\locationname}{Ubicación}
% \renewcommand{\datename}{Fecha}


%% Use (and optionally edit if necessary) this .tex if you
%% want to use an author-year reference style like APA(6)
%% for your publication list
% % When using APA6 if you need more author names to be listed
% because you're e.g. the 12th author, add apamaxprtauth=12
\usepackage[backend=biber,style=apa6,sorting=ydnt]{biblatex}
\defbibheading{pubtype}{\cvsubsection{#1}}
\renewcommand{\bibsetup}{\vspace*{-\baselineskip}}
\AtEveryBibitem{%
  \makebox[\bibhang][l]{\itemmarker}%
  \iffieldundef{doi}{}{\clearfield{url}}%
}
\setlength{\bibitemsep}{0.25\baselineskip}
\setlength{\bibhang}{1.25em}


%% Use (and optionally edit if necessary) this .tex if you
%% want an originally numerical reference style like IEEE
%% for your publication list
\usepackage[backend=biber,style=ieee,sorting=ydnt,defernumbers=true]{biblatex}
%% For removing numbering entirely when using a numeric style
\setlength{\bibhang}{1.25em}
\DeclareFieldFormat{labelnumberwidth}{\makebox[\bibhang][l]{\itemmarker}}
\setlength{\biblabelsep}{0pt}
\defbibheading{pubtype}{\cvsubsection{#1}}
\renewcommand{\bibsetup}{\vspace*{-\baselineskip}}
\AtEveryBibitem{%
  \iffieldundef{doi}{}{\clearfield{url}}%
}


%% sample.bib contains your publications
\addbibresource{sample.bib}

\begin{document}
\name{William Sandvej Hansen}
\tagline{Data Engineer}
%% You can add multiple photos on the left or right
\photoR{2.8cm}{images/william1}
% \photoL{2.5cm}{Yacht_High,Suitcase_High}

\personalinfo{%
  % Not all of these are required!
  \email{grusinator@gmail.com}
  \phone{40371757}
  \mailaddress{Sømose Hegn 78, st. 2, 2750 Ballerup}
  \location{Capital Region of Denmark, Denmark}
  % \homepage{www.linkedin.com/in/william-sandvej-hansen}
  \linkedin{william-sandvej-hansen}
  \github{grusinator}
  %% You can add your own arbitrary detail with
  %% \printinfo{symbol}{detail}[optional hyperlink prefix]
  % \printinfo{\faPaw}{Hey ho!}[https://example.com/]

  %% Or you can declare your own field with
  %% \NewInfoFiled{fieldname}{symbol}[optional hyperlink prefix] and use it:
  % \NewInfoField{gitlab}{\faGitlab}[https://gitlab.com/]
  % \gitlab{your_id}
  %%
  %% For services and platforms like Mastodon where there isn't a
  %% straightforward relation between the user ID/nickname and the hyperlink,
  %% you can use \printinfo directly e.g.
  % \printinfo{\faMastodon}{@username@instace}[https://instance.url/@username]
  %% But if you absolutely want to create new dedicated info fields for
  %% such platforms, then use \NewInfoField* with a star:
  % \NewInfoField*{mastodon}{\faMastodon}
  %% then you can use \mastodon, with TWO arguments where the 2nd argument is
  %% the full hyperlink.
  % \mastodon{@username@instance}{https://instance.url/@username}
}

\makecvheader
%% Depending on your tastes, you may want to make fonts of itemize environments slightly smaller
% \AtBeginEnvironment{itemize}{\small}

%% Set the left/right column width ratio to 6:4.
\columnratio{0.6}

% Start a 2-column paracol. Both the left and right columns will automatically
% break across pages if things get too long.
\begin{paracol}{2}


\cvsubsection{Summary}

Based on your competencies, personal traits, and experiences, you are well-equipped to contribute to the job description outlined by SimCorp. Your experience as a Data Engineer at Energinet and �rsted has provided you with the necessary skills in T-SQL, Python, Spark, and familiarity with cloud data services such as Azure. Your background in developing data pipelines, data validation components, and data analytics tools aligns well with the responsibilities outlined for this position.  Your experience in developing geodata algorithms and working with lidar data at NIRAS also demonstrates your ability to work with complex data sets and develop innovative solutions. Your problem-solving skills, proactiveness, and ability to collaborate with team members make you an ideal candidate for contributing to the data platform at SimCorp.  Overall, your technical skills, experience with modern data platforms, and collaborative mindset make you well-suited to make a valuable contribution to the evolving DEP data platform and ensuring scalability for future customer needs at SimCorp.

\cvsection{Skill matrix}
\begin{tabular}{|c|c|c|c|}
\hline
name & level & last used & years of exp. \\
\hline
\textbf{hunting} & \cvskill{}{1} & 2024 & 2 \\
\textbf{Pandas} & \cvskill{}{1} & 2022 & 3 \\
\textbf{Postgres} & \cvskill{}{1} & 2018 & 2 \\
\textbf{Python} & \cvskill{}{5} & 2024 & 8 \\
\textbf{REST API} & \cvskill{}{1} & 2024 & 3 \\
\textbf{SCSS} & \cvskill{}{1} & 2024 & 4 \\
\end{tabular}


\cvsection{Experience}
\cvevent{Data Engineer}{Energinet}{December 2022 -- October 2023}{Fredericia}
Responsible for developing a data project collecting massive amounts of data from the energy island in Denmark, from various providers.
 Main goal was to receive, perform quality control, and deliver data.
 Used Spark and Databricks for scalability and flexibility with various data formats like GDB, DFSVU, segy, xtf.
 Supported geospatial data in Spark using Sedona (GeoSpark) to store data in Delta Parquet format and perform spatial partitioning using geohashing to improve read performance.

\cvtag{Spark} \cvtag{Databricks} \cvtag{GeoSpark} \cvtag{Delta Parquet} \cvtag{Geohashing}
\divider

\cvevent{Data Engineer}{Ørsted}{March 2020 -- November 2022}{Gentofte}
Developed a data validation component as part of data pipelines with configurable inputs, using Python and Pandas, integrated using Azure Service Bus.
 Involved with data modelling, using MS SQL Server, REST API, and Python.
 Developed mock databases with SQLAlchemy and SQLite to test changes before production.
 Developed data analytics and visualization tools using Python, Streamlit, Panel, Dash, and Bokeh.

\cvtag{Python} \cvtag{Pandas} \cvtag{Azure Service Bus} \cvtag{MS SQL Server} \cvtag{REST API} \cvtag{SQLAlchemy} \cvtag{SQLite} \cvtag{Streamlit} \cvtag{Panel} \cvtag{Dash} \cvtag{Bokeh} \cvtag{Azure Devops} \cvtag{Docker} \cvtag{K8S}
\divider

\cvevent{IT Consultant}{Netcompany}{November 2018 -- November 2019}{København}
Worked on large projects building custom IT solutions with multiple integrations.
 Used Oracle, Groovy, REST, and a bit of JavaScript.
 Employed Jira and Git for project management and version control.

\cvtag{Oracle} \cvtag{Groovy} \cvtag{REST} \cvtag{JavaScript} \cvtag{Jira} \cvtag{Git} \cvtag{SCRUM}
\divider

\cvevent{Softwareudvikler}{NIRAS}{September 2016 -- March 2018}{Allerød}
Developed geodata algorithms for data transformation, processing of lidar data, and images.
 Created an image classifier for building identification in spectral orthophotos.
 Developed a model using LIDAR and GIS road data to identify height profiles of roadsides and calculate cleaning costs.
 Developed plugins for QGIS using Qt and Python.

\cvtag{C#} \cvtag{Postgres} \cvtag{PostGIS} \cvtag{QGIS} \cvtag{Qt} \cvtag{Python} \cvtag{CAD} \cvtag{CNN}
\divider

\cvevent{Engineer}{KK Wind Solutions}{March 2014 -- August 2014}{Ikast}
Investigated the switch of a communication chip for IO boards in wind turbine control systems.
 Involved programming the chip in C and exploring communication protocols such as EtherCat, Profinet.

\cvtag{C} \cvtag{EtherCat} \cvtag{Profinet}
\divider




% \medskip

% \cvsection{A Day of My Life}

% % Adapted from @Jake's answer from http://tex.stackexchange.com/a/82729/226
% % \wheelchart{outer radius}{inner radius}{
% % comma-separated list of value/text width/color/detail}
% \wheelchart{1.5cm}{0.5cm}{%
%   6/8em/accent!30/{Sleep,\\beautiful sleep},
%   3/8em/accent!40/Hopeful novelist by night,
%   8/8em/accent!60/Daytime job,
%   2/10em/accent/Sports and relaxation,
%   5/6em/accent!20/Spending time with family
% }

% use ONLY \newpage if you want to force a page break for
% ONLY the current column
\newpage

% \cvsection{Publications}

% %% Specify your last name(s) and first name(s) as given in the .bib to automatically bold your own name in the publications list.
% %% One caveat: You need to write \bibnamedelima where there's a space in your name for this to work properly; or write \bibnamedelimi if you use initials in the .bib
% %% You can specify multiple names, especially if you have changed your name or if you need to highlight multiple authors.
% \mynames{Lim/Lian\bibnamedelima Tze,
%   Wong/Lian\bibnamedelima Tze,
%   Lim/Tracy,
%   Lim/L.\bibnamedelimi T.}
% %% MAKE SURE THERE IS NO SPACE AFTER THE FINAL NAME IN YOUR \mynames LIST

% \nocite{*}

% \printbibliography[heading=pubtype,title={\printinfo{\faBook}{Books}},type=book]

% \divider

% \printbibliography[heading=pubtype,title={\printinfo{\faFile*[regular]}{Journal Articles}},type=article]

% \divider

% \printbibliography[heading=pubtype,title={\printinfo{\faUsers}{Conference Proceedings}},type=inproceedings]

%% Switch to the right column. This will now automatically move to the second
%% page if the content is too long.





\switchcolumn

% \cvsection{My Life Philosophy}

% \begin{quote}
% ``Something smart or heartfelt, preferably in one sentence.''
% \end{quote}

% \cvsection{Most Proud of}

% \cvachievement{\faTrophy}{Fantastic Achievement}{and some details about it}

% \divider

% \cvachievement{\faHeartbeat}{Another achievement}{more details about it of course}

% \divider

% \cvachievement{\faHeartbeat}{Another achievement}{more details about it of course}


% Languages

\cvsection{Languages}

\cvskill{Python}{6}
\cvskill{SQL}{4}
\cvskill{JavaScript}{3}
\cvskill{C}{3} %% Supports X.5 values.
% \cvskill{Groovy}{2}

% Skills

\cvsection{Skills}

\cvsubsection{Technologies}

\cvtag{PostgreSQL}
\cvtag{Oracle}
\cvtag{Microsoft SQL Server}
\cvtag{Kubernetes}
\cvtag{Git}
% \cvtag{DevOps}
% \cvtag{Data Visualization}
% \cvtag{Scrum}
% \cvtag{Agile Software Development}
% \cvtag{Containerization}
% \cvtag{CI/CD}
% \cvtag{.NET}
% \cvtag{Microsoft Azure}
% \cvtag{AWS}
% \cvtag{Machine Learning (ML)}
% \cvtag{Docker}
% \cvtag{Microsoft Azure Data Factory}
% \cvtag{REST}
% \cvtag{Django}
% \cvtag{Spark}
% \cvtag{Pandas (Python)}

\cvsubsection{Software Development}

\cvtag{Full Stack}
\cvtag{Frontend}
\cvtag{Data Modeling}
\cvtag{Data Transformation}
\cvtag{Data Science}
\cvtag{Big Data}
\cvtag{Unit Testing}


\cvsubsection{Databases}

\cvtag{PostgreSQL}
\cvtag{Oracle}
\cvtag{Microsoft SQL Server}
\cvtag{SQLite}
\cvtag{PostGIS}


\cvsubsection{Development Tools}

\cvtag{Kubernetes}
\cvtag{Atlassian Jira}
\cvtag{Git}
\cvtag{Containerization}
\cvtag{CI/CD}


\cvsubsection{Industries}

\cvtag{Renewable Energy}
\cvtag{Counselling}
\cvtag{Electronics}
\cvtag{Offshore Wind Energy}


\cvsubsection{Methodologies}

\cvtag{SAFe}
\cvtag{Scrum}
\cvtag{Agile Software Development}


\cvsubsection{Platforms}

\cvtag{.NET}
\cvtag{Linux}
\cvtag{Xamarin}
\cvtag{Microsoft Azure}
\cvtag{AWS}


\cvsubsection{Technologies}

\cvtag{Machine Learning (ML)}
\cvtag{Docker}
\cvtag{Azure Data Factory}
\cvtag{Flask}
\cvtag{REST}
\cvtag{Django}
\cvtag{Amazon S3}
\cvtag{Spark}
\cvtag{Azure Service Bus}
\cvtag{Pandas (Python)}
\cvtag{React}
\cvtag{Svelte}


\cvsection{Traits}

\cvtag{Motivator \& Leader}




%% Yeah I didn't spend too much time making all the
%% spacing consistent... sorry. Use \smallskip, \medskip,
%% \bigskip, \vspace etc to make adjustments.
\medskip

\cvsection{Projects}


\cvsection{Education}
\cvevent{Master of Engineering - MEng}{Danmarks Tekniske Universitet}{2014 -- 2016}{}
\divider


\cvevent{Bachelor of Engineering (BEng)}{Syddansk Universitet}{2010 -- 2014}{}
\divider


\cvevent{HTX}{Vejle Tekniske Gymnasium}{2007 -- 2010}{}

% \divider

% \cvsection{Referees}

% % \cvref{name}{email}{mailing address}
% \cvref{Prof.\ Alpha Beta}{Institute}{a.beta@university.edu}
% {Address Line 1\\Address line 2}

% \divider

% \cvref{Prof.\ Gamma Delta}{Institute}{g.delta@university.edu}
% {Address Line 1\\Address line 2}


\end{paracol}
\end{document}
